\documentclass[12pt,a4paper]{article}
\usepackage{minionpro}
\newcommand{\ud}{\mathrm{d}}
\usepackage[hmargin=2cm,vmargin=2cm]{geometry}
\setcounter{secnumdepth}{0}

\begin{document}

\section{Tilman's R$^*$ model}

I'm going to break with the notation above and work with notation that is more closely aligned with that in the Tilman papers.  I'll draw the notations together afterwards.

Let $N_i$ ($i = 1, \ldots, n$) be the density of the $i$th species, of which there are $n$, and let $R_j$ ($j = 1, \ldots, k$) be the abundance of the $j$th resource, of which there are $k$.

The basic equations are (general form in Tilaman 1980 and 1982):

\begin{equation}
\begin{split}
\frac{\ud N_i}{\ud t} =& N_i (f_i(R_1, \ldots, R_k) - m_i)\\
\frac{\ud R_j}{\ud t} =& g_j(R_j) - \sum_i N_i f_i(R_1, \ldots, R_k) h_{ij}(R_1, \ldots, R_k)
\end{split}
\end{equation}
%
where $m_i$ is the mortality rate of the $i$th species, $f_i$ is the
functional relationship between resource availability and per-capita
population change for species $i$, $g_j$ is a function describing the
process of resupply of resource $j$, $h_{ij}$ is a function describing
the amount of resource $j$ required to produce each new individual of
species $i$.

Typically, $g_j$ is $D_j (S_j - R_j)$, where $D_j$ is the
\emph{turnover rate} of the $j$th resource (and typically $D_j = D$)
and $S_j$ is the \emph{supply rate} of resource $j$.

For essential resources, which is what we will focus on first, the
$f_i$ can be replaced with $\min_j[\rho_{ij}(R_j)]$ (following Huisman
2001).  The function $\rho_{ij}$ is the specific growth rate on
resource $j$ of species $i$ and is typically given by the Monod
equation
%
\begin{equation*}
  \rho_{ij}(R_j) = \frac{r_i R_j}{K_{ij} + R_j}
\end{equation*}
%
where $r_i$ is the maximum specific growth rate of species $i$ and
$K_{ij}$ is the half-saturation constant for resource $j$ of species
$i$.  In what follows below we'll also use the inverse function
\begin{equation*}
  \rho_{ij}^{-1}(g_i) = \frac{K_{ij}g_i}{r_i - g_i}
\end{equation*}
which gives the amount of resouce $j$ required to produce the specific
growth rate $g_i$ of species $i$.

Huisman also writes the $h_{ij}$ function as the constant $C_{ij}$,
which is the rate of consumption of the $j$th resource by species
$i$.

\subsection{Things we need}

\subsubsection{Maximum growth rate}

There are a couple of different interpretations here.  First there is
the unconditional maximum growth rate, which is $r_i$ --- this is the
highest growth rate that is possible under any situation given htat
$R_j/(K_{ij} + R_{j})$ has a maximum of 1 as $R_j \to \infty$ (if we
had a function $f_i$ that summed the per-resource components we'd need
to be careful here).

The other interpretation is to use maximum resource level that a
species might encounter, which will be $S_j$, so the maximum growth
rate becomes

\begin{equation*}
\min_j\left[\rho_{ij}(S_j)\right] =
\min_j\left[\frac{r_i S_j}{K_{ij} + S_{j}}\right]
\end{equation*}

\subsubsection{Carrying capacity}

I'm defining this as the population size when the species is in a
monoculture (so $N_l = 0$ for $l \neq i$).  If $j$ is the limiting
resource (i.e. $\min_l[\rho_{il}] = \rho_{ij}$), then because $\ud N_i / \ud
t = 0$

\begin{equation*}
  \rho_{ij}(R_j) = m_i \quad\Rightarrow\quad
  \hat R_{ij} = \rho_{ij}^{-1}(m_i) = \frac{m_i K_{ij}}{r_i - m_i}
\end{equation*}

We don't actually need this to get the equilibrium population size.
Because at equilibrium $m_i = f_i(R_1, \ldots, R_k)$ and $\ud R_j /
\ud t = 0$, and because we're interested in the case where only a
single species is present

\begin{equation*}
  g_j(R_j) = N_i m_i h_{ij}(R_1, \ldots, R_k) \quad\Rightarrow\quad
  N_i = \frac{g_j(R_j)}{m_i h_{ij}(R_1, \ldots, R_k)}
\end{equation*}

For our specific version of the model, this is
\begin{equation*}
  \hat N_i = K_i = \frac{D_j (S_j - R_j)}{m_i C_{ij}}
\end{equation*}

(note that this approach does not give us the equilibrium values for
all resources.  Looking at the code there is a bit of care required
here).



\end{document}

